\documentclass[../main.tex]{subfiles}
\begin{document}
\section{FIS/01}
\subsection{Particle Physics\\(Già Fisica nucleare e subnucleare II)}
\subsubsection{Paolo Bagnaia}
\paragraph{Modalità:}
\begin{itemize}
    \item Argomento a piacere (15 minuti con domande sull'argomento) + 1/2 domande sul programma d'esame.
    \item L'argomento a piacere é da concordare in anticipo col professore e può essere l'approfondimento di un argomento trattato a lezione, come un esperimento o altro.
\end{itemize}
\paragraph{Domande}
*Elenco incompleto basato su quanto detto nei vari gruppi Whatsapp*
\begin{itemize}
    \item Fattor i di forma (\textit{Form Factors})
    \item Parlare del DIS (Urto Profondamente Anelastico - \textit{Deep Inelastic Scattering})
    \item Perché $J/\psi$ ha un'ampiezza di decadimento così stretta (Regola di Zweig (OZI))
    \item Mesoni K
    \item Funzione d'onda $\Delta^{++}$
    \item Evidenza sperimentale dell'introduzione del numero quantico di colore\\
    \textit{Il perché del colore, disegnare un particolare multipletto}
    \item Si concentra molto sulla domanda che porti a scelta. Ho portato i mesoni K e mi ha chiesto: 

Perché se due particelle sono l'una l'antiparticella dell'altra hanno stessa massa? (conservazione CPT). 

In 1 m di materiale quando $K_{\text{long}}$ decadono? Quante $K_{short}$?

I Cherenkov possono sbagliare nell'esperimento della conservazione CP, in che modo? I Cherenkov dovrebbero emettere luce al passaggio dell'elettrone e non emetterla al passaggio del pione, ma l'emissione è un fenomeno quantistico e la luce emessa potrebbe essere talmente fioca da non essere rilevata, allora scambio un elettrone con un pione.

Come sondi la struttura del protone? Domanda molto generale, non si è concentrato sui dettagli.
\end{itemize}
\paragraph{03/11/2022}
\begin{enumerate}
    \item Io ho portato l'esperimento di Hofstadter e mi ha chiesto qual erano le quantità interessati a misurare e perché (oltre alla sezione d'urto, l'energia degli elettroni per due motivi 1) calibrare lo strumento 2) discriminare gli eventi di scattering elastico da quello inelastico), poi mi ha chiesto perché $G_m$ del protone vale 2,79 e perché a energie più alte dipende da $Q^2$. Mentre l'altro tizio (quello della commissione) mi ha chiesto perché $J/\psi$ ha un'ampiezza di decadimento così stretta (regola OZI)
\end{enumerate}
\subsection{Medical applications of physics}
\subsubsection{Naurang Lal Saini}
\paragraph{Esperienze}
\begin{itemize}
    \item  Ho fatto l'esame di Medical application of physics con Saini. Il professore è molto tranquillo e l'esame tende a durare oltre l'ora e mezza. Copre tutto il programma e queste sono le domande che mi ha fatto, nel caso possano essere utili a qualcuno. Mi ha detto di ricordare a chi lo deve sostenere che fare un orale che sfiora le due ore è molto pesante e quindi bisogna prepararsi mentalmente e fisicamente. Ah, la prima cosa che chiede é il curriculum che si sta seguendo e la domanda ha un forte impatto sulle domande d'esame e sulla valutazione
\end{itemize}
\paragraph{Aprile 2022}
Tutto questo è un solo orale
\begin{itemize}
    \item Prima parte: 
    \begin{enumerate}
        \item In che modo interagiscono i neutroni ?
        \item In che modo interagiscono i fotoni ? Con Grafico degli effetti dominanti
        \item Grafico sezioni d'urto fotoni
        \item Perché usiamo neutroni e non raggi x per l'analisi delle strutture ?
        \item Ionizza di più la radiazione Alfa o beta?  Perchè ?
        \item Calcolo ionizzazione specifica alfa data energia 
        \item picco di bragg. Come lo usiamo in terapia ? Come varia al variare dell'energia ? E in confronto ad altre particelle ?
        \item coefficiente di attenuazione, definizione e grafico 
        \item spettro del bremmstr.
        \item effetto heel 
        \item In quale parte dell'anodo colpiscono gli elettroni ?
        \item perché l'anodo ha quella forma ?
    \end{enumerate}
    \item Radiografia e tomografia
    \begin{enumerate}
        \item curva HD e parametri collegati. Come varia e le differenze di grafico tra classica e digitale 
        \item densità ottica di fondo 
        \item artefatti della tomografia
        -tipo di ricostruzione per la ct a spirale
        -MSAD e fantoccio 
    \end{enumerate}
    \item MEDICINA E IMAGING NUCLEARE
    \begin{enumerate}
        \item generatore Moly
        \item Necr e sue applicazioni pratiche per la risoluzione
        \item In cosa consiste in pratica la pet 
    \end{enumerate}
    \item ULTRASUONI
    \begin{enumerate}
        \item  Lobi reticolari 
        \item come miglioro la risoluzione assiale (o laterale, non mi ricordo quale delle due ha chiesto)
        \item CMUT
\item imaging armonico
\item TGC
\end{enumerate}
\item RISONANZA MAGNETICA
\begin{enumerate}
\item cosa sono i gradienti e a che servono
\item angolo di Ernst
\item Andamento di Mz e Mxy 
\item Tra T1, T2, e T2*, chi è più grande di chi ?
\end{enumerate}
\end{itemize}
\newpage
\subsection{Neutrini e Materia Oscura\\
\textit{Neutrinos and dark matter}}
\subsubsection{Marco Vignati}
\paragraph{23 Dicembre 2022}
\begin{itemize}
    \item Allora a me ha chiesto inizialmente il seminario, tipo domanda a piacere, e poi mi ha fatto una domanda di teoria sui neutrini (visto che il seminario in classe era sulla materia oscura e quello che ho portato sugli assioni), voleva il calcolo con l'oscillazione dei neutrini nella materia fino a commentare con quale autostato della massa vengono emessi i neutrini dal sole e con quale probabilità vengono rivelati come neutrini elettronici\\
    Seminario durante anno: PADME\\
    Seminario esame: ALPS\\
    Voto: 28
\end{itemize}
\newpage
\subsection{Laboratorio di Fisica I}
\subsubsection{Gianluca Cavoto (HEP-PART)}
\paragraph{Esperienze}
\begin{itemize}
    \item Fa sempre due domande.
    \item Io ho risposto abbastanza bene alla prima domanda e alla seconda sono caduta su una serie di stronzate e mi ha messo 28.
    \item Io ero un pochino agitato e credo si sia sentito molto dal modo in cui ho articolato il discorso. In ogni caso, nonostante qualche esitazione, ho preso 30. Le domande che mi sono state fatte erano molto più fattibili degli esercizi che abbiamo sulle dispense. In ogni caso, è importante ricordare a memoria delle quantità tipiche di alcuni materiali/detector e saper fare dei conti al volo. Per quanto riguarda il mio esame, era necessario saper discutere di un materiale scintillante. Ho scelto lo Ioduro di Sodio. Di questo occorre saper stimare la lunghezza di radiazione (ordine del centimetro) e l'energia critica. Dato un fotone in ingresso da 100 Mev, bisogna stimare la lunghezza del calorimetro. Mi è stato utile, poi, ricordare la risoluzione tipica di una camera proporzionale a molti fili.
    \item \texttt{Febbraio 2022} È molto tranquillo, ti dà il tempo necessario per farti pensare, considerando anche che alcune domande potrebbero essere un po' meno immediate di altre. Consiglio di preparare così laboratorio: iniziare con la teoria, concentrandosi sulle configurazioni sperimentali soprattutto (non chiederà mai invece dimostrazioni tipo quella di Bethe e Bloch). È inoltre molto buono sapere a memoria alcune proprietà dei materiali (densità, lunghezze di radiazione, indici di rifrazioni, etc). Fatta la teoria, consiglio di fare gli esercizi svolti da lui in classe, che tende a chiedere, e sono un'ottima base da cui partire.
\end{itemize}
Riassunto\\
\texttt{Con questo tipo di carattere indico le lezioni dell'anno accademico 2020/2021}
\begin{enumerate}
    \item A me ha detto di immaginare di voler misurare l'attività di una sorgente composta da un radionuclide, il sodio 22, $^{22}$Na, aggiungendo che decade $\beta^+$. Il positrone sta dentro la sorgente stessa e possiede una certa energia. Cosa esce dalla sorgente secondo lei? Supponi la sorgente sia molto spessa, un blocchetto immerso in un materiale tale che il range del positrone rimanga dentro. \texttt{(Lezione 2,17)};
    \item Supponga di avere una sorgente $\beta$ che emette su tutto l'angolo solido, ma in maniera non uniforme. Definiamo tanto per cominciare un sistema di riferimento $xy$, la sorgente sta nell'origine e chiamiamo $\theta$ l'angolo rispetto all'origine e dobbiamo misurare la distribuzione angolare di queste particelle $\beta$. In realtà sappiamo che questa distribuzione angolare ha un andamento proporzionale a $\cos\theta$. Per cui lo spettro di questi $\beta$ è circa \(\frac{\textrm{d}N}{\textrm{d}\cos\theta}\sim \cos\theta \). Come costruirebbe un rivelatore capace di misurare questa distribuzione angolare? Intanto mi dica cosa vuol dire particella $\beta$. \\
    \textit{Disegna lo spettro della particella $\beta$. Quant'è il Q valore? (risp: circa 1 Mev)}
    \item Per poter misurare il tempo di vita medio, o \textit{lifetime} o tempo di decadimento, dei muoni che sono nei raggi cosmici con un'incertezza del per cento, che tipo di sistema/rivelatore userei, che sia il più semplice possibile? E quanto tempo dovrei aspettare prima di avere una misura con questo livello di incertezza? Assumi di trovarti al livello del mare e che i raggi cosmici abbiano quantità di moto pari a 1 GeV \texttt{(Lezione 3, 10)};\\
    \textit{Che tipo di rilevatori useresti , come li posizioneresti, qual è il tasso di raggi cosmici ... Le domande non sono molto di tipo teorico}
    \item Come misurare il decadimento di un pione neutro di 1 GeV che decade in due fotoni: \(\pi^0\to\gamma\gamma\) \texttt{(Lezione 3, lezione 23, spiegazione esercizio 1.6)};
    \item Incertezza su angoli per disposizione a telescopio \texttt{(Lezione 7 e inizio 8)}
    \item Diffusione coulombiana multipla (\textit{multiple Coulombian scattering}) (Molière) \texttt{(Lezione 7)};
    \item come misurare efficienza di un fotone in un contenitore con un gas \texttt{(Lezione 9,10)};\\
    \textit{Intende prendere la probabilità che un fotone percorra una distanza x , cioè  $e^{-\mu x}$. L'efficienza è 1 - questa probabilità}
    \item Camere a deriva (\textit{Drift chambers}) \texttt{(Lezione 13)};
    \item Camera a proiezione temporale (TPC); \texttt{(Lezione 13,14)};\\
    \textit{Principi di funzionamento}
    \item Misurare l'impulso di un muone con un campo magnetico. \texttt{(Lezione 3,17)}
    \item Fascio di muoni a 10 GeV, come ne misuro il tempo di vita medio (la \textit{lifetime}); rivelatore muonico \texttt{(Lezione 17)}; 
    \item Fascio di adroni (pioni $\pi$, kaoni $K$, protoni $p$) a 100 o 50 GeV, rivelatore per distinguerli \texttt{(Lezione 17)};\\
    \textit{Quanti rivelatori, la loro posizione, che materiale, quanto vale l'angolo Č dopo che il pione è passato per il primo rivelatore.\\ Ho scritto le masse di ciascuna particella e la mia idea era quella di usare tre rivelatori Čerenkov; ho calcolato l'indice di rifrazione minimo per aver luce Čerenkov (mi ha fermato dopo aver calcolato quello del pione), il punto era che usando gas l'angolo di emissione era molto piccolo (ho scritto la formula) e i fotoni molto pochi (ho scritto la formula), non ho capito come risolvere questo problema, ho citato di mettere degli specchi ma non so se non gli andava bene o non ha sentito o non l'ho sottolineato abbastanza)}
    \item Come misuro l’energia di un fotone ad 1 MeV e i rivelatori di fotoni \texttt{(Lezione 16,18)};
    \item Ah chiede sempre quella storia di come evitare che i fotoni Compton scappino dal rilevatore coi conti \texttt{(Lezione 16-18)};
    \item Come rivelare la direzione di un fotone di 100 MeV \texttt{(Lezione 18)}.
    \item Contatore Čerenkov \texttt{(Lezione 8,15,19)};
    \item Čerenkov per distinguere particelle in fascio monocromatico \texttt{(Lezione 8,15,19)};
    \item Raccontami come funziona un fotodiodo (\texttt{Lezione 20});\\\textit{differenza fotomoltiplicatore e fotodiodo}
    \item Supponiamo di voler misurare la divergenza (quindi la distribuzione degli angoli) di un fascio di muoni che hanno quantità di moto 100 GeV e che viaggiano nel vuoto in un tubo. Ogni muone ha una direzione diversa riszpetto all'asse orizzontale. Quindi per ogni particella bisogna misurare il suo angolo, come farebbe lei?>>
    \item Supponga di avere un esperimento in cui ci sono particelle cariche stabili, per cui ci sono elettroni, muoni, pioni, kaoni e protoni e poi di avere un rivelatóre a luce Čerenkov, fatto di \textit{fused silica}, che dovrebbe sapere cos'è: è un materiale trasparente che ha un indice di rifrazione di 1,5. Mi sa disegnare per queste 5 specie la relazione che c'è tra l'angolo Čerenkov e l'impulso? L'intervallo dell'impulso di queste particelle è tra 100 MeV e 3 GeV e ognuna ha impulso diverso. Vorrei che mi disegnasse 5 curve mettendo sull'asse delle $x$ l'impulso e sull'asse delle $y$ l'angolo Čerenkov \texttt{(Lezione 8,15,19)}; 
    \item relazione tra carica e potenziale in un gas detector \texttt{(Lezione 11,13,21)};\\
    \textit{regime di proporzionalità, regime geiger, quel grafico lì.}
    \item Tubo fotomultiplicatore (PMT) \texttt{(Lezione 19-21)};\\
    \textit{cosa sono, gain, efficienza, per cosa si usa}
    \item Camera a molti fili (\texttt{Lezione 22})\\
    \textit{In una "wire chamber" vengono prodotti tanti elettroni, tanta carica, poi questo induce un segnali sui fili. Lei mi saprebbe dire come si calcola il segnale elettrico indotto su un elettrodo da una carica in moto? Risposta: teorema di Ramo\\
    Quanti elettroni vengono prodotti dal passaggio della particella?\\
    Qual è la risoluzione?}
    \item Discutere di un materiale scintillante \texttt{(Lezione 18-20,22)}.\\ {\it Ho scelto lo Ioduro di Sodio NaI. Di questo occorre saper stimare la lunghezza di radiazione (ordine del centimetro) e l'energia critica. Dato un fotone in ingresso da 100 Mev, bisogna stimare la lunghezza del calorimetro. Mi è stato utile, poi, ricordare la risoluzione tipica di una camera proporzionale a molti fili};
    \item Ho un fotone a 100 MeV come ne misuro l'energia e la direzione \texttt{(Lezione 19-22)};\\
    \textit{discussioni varie su calorimetro e PMT; quanto è l' efficienza quantica, quanto deve essere lungo il calorimetro, errore sull'energia.}
    \item l'assistente (tale Cecilia) mi ha chiesto come distinguere pioni e kaoni con impulso a 1 GeV, ovviamente voleva sapere materiali e dimensioni di tutto.
    \item Grafico sezione d'urto fotoni;
    \item Assistente [Bauce]: immagina di voler ricostruire posizione (e quindi l'angolo) con cui arrivano dei muoni nei raggi cosmici, che sistema useresti?\\ 
{\it Io ho detto degli scintillatori con dei PMT alla fine, come in una diapositiva mostrata a lezione, mi ha chiesto la velocità effettiva della luce nel mezzo (circa la metà della velocità della luce nel vuoto: 15 cm/ns), quanto erano grandi questi scintillatori (ho detto 10 cm, ma non so se gli andava bene, non penso) e la stima dell'errore sulla posizione (ho assunto trascurabile l'incertezza sulla velocità effettiva e 1 ns quella sul tempo, il risultato era di circa 11 cm). L'assistente mi ha chiesto se volessi davvero conoscere l'errore sulla velocità effettiva cosa farei (Bauci ha detto avrebbe fatto una calibrazione prima, io avevo capito dovessi usare sempre i muoni dei raggi cosmici).}
   \item Come riuscire a distinguere muoni positivi da muoni negativi entrambi con un impulso di 10 GeV\\
    \textit{Quanto devo fare grosso il mio rivelatore, la dimensione degli scintillatori dopo}
    \item Quali sono le dimensioni di uno scintillatore composto da NaI perché si abbia contenimento al 95\% di un fascio di raggi gamma da 1 MeV?
\end{enumerate}
\paragraph{21 Gennaio 2021}
    \begin{enumerate}
        \item Ho un fotone a 100 MeV come ne misuro l'energia e la direzione.\\
        \textit{discussioni varie su calorimetro e PMT; quanto è la quantum efficiency, quanto deve essere lungo il calorimetro, errore sull'energia.}
        \item Misurare l'impulso di un muone con un campo magnetico. 
    \end{enumerate}
    \begin{enumerate}
        \item Fascio di muoni a 10 GeV, come ne misuro la lifetime; rivelatore muonico. 
        \item Relazione tra carica e potenziale in un gas detector;\\
        \textit{regime di proporzionalità, regime geiger, quel grafico lì.}
    \end{enumerate}
    \begin{enumerate}
        \item Fascio di Adroni (pioni, kaoni, protoni) a 100 GeV, rivelatore per distinguerli.\\
        $\vdots$\\
        \textit{Il candidato ha risposto <<posiziono 3 rivelatori perché, se scelgo bene gli indici di rifrazione, una particella attiverà tutti e tre i rivelatori, una solo due e una solo uno>>. Cavoto gli ha chiesto quali particelle attivano quali rivelatori. Lei ha risposto: il pione essendo la più leggera arriverà fino in fondo, il kaone fino al secondo. Cavoto gli ha chiesto di trovare gli indici di rifrazione
        \[
        n<\frac{1}{1-\frac{1}{2}\frac{m^2}{10^4\textrm{ GeV$^2$}}}
        \]
        per il pione
        \[
        n<\frac{1}{1-\frac{1}{2}\frac{10^4 \textrm{MeV$^2$}}{10^4\cdot10^6\textrm{ MeV$^2$}}}=\frac{1}{1-\frac{1}{2}10^{-6}}\approx 1 + 0.5\cdot10^{-6}
        \]
        Cavoto: <<Come si fa a realizzare un rivelatore con questo indice di rifrazione? Con quale materiale?>>
        \begin{flushright}
        <<L'aereogel è molto vicino a 1>>:R
        \end{flushright}
        <<Si ricorda quanto?>>
        \begin{flushright}
        \(n-1\sim 10^{-3}\)
        \end{flushright}
        <<E va bene in questo caso?>>
        \begin{flushright}
        <<No, è troppo grande>>
        \end{flushright}
        <<Un materiale con indice più piccolo?>>
        \begin{flushright}
        <<L'aria>>
        \end{flushright}
        <<Meglio un gas puro, come l'elio. Si ricorda come si può cambiare l'indice di rifrazione agendo su un parametro macroscopico del gas?>>
        \begin{flushright}
        <<Agendo sulla densità, che per l'elio \(\rho_{\textrm{He}}\sim 10^{-4}\)>>
        \end{flushright}
        <<Supponi di avere un volume pieno di gas, come cambio la densità? Cambiando la pressione. A pressione più bassa cosa ottiene?>>
        \begin{flushright}
        <<Un gas più rarefatto>>
        \end{flushright}
        <<L'indice di rifrazione come va? Se dovesse azzardare un andamento.>>
        \begin{flushright}
        <<Più sarà denso più avrà un indice di rifrazione alto>>
        \end{flushright}
        <<Supponi di aver trovato questo gas, mi sa disegnare il rivelatore? Come se lo immagina?>>
        \begin{flushright}
        *Disegna tre parallelepipedi in posizione verticale* 
        \end{flushright}
        <<Discutiamone solo uno, gli altri saranno uguali, quale sarebbe il suo spessore?>>
        \begin{flushright}
        <<La particella arriva, emette il fotone Č...>>
        \end{flushright}
        <<Con quale angolo lo emette?>>
        \begin{flushright}
        \(\cos{\theta_c}=\frac{1}{\beta n} \ \rightarrow \ \theta_c=\arccos{\frac{1}{\beta n}}\)
        \end{flushright}
        <<Nel nostro caso quanto vale?>>
        \begin{flushright}
        \(\beta \approx 1 \ \rightarrow \  \theta_c=\arccos{\frac{1}{1+\frac{1}{2}10^{-6}}}\)\\
        L'angolo sarà molto piccolo.
        \end{flushright}
        <<Come fai a calcolarlo? In radianti.>>
        \begin{flushright}
        Espando il coseno:\\
        \(1-\frac{\theta_c^2}{2}=\frac{1}{n} \ \rightarrow \ \theta_c=\sqrt{2\left(1-\frac{1}{1+10^{-6}}\right)}=\sqrt{2\frac{10^{-6}}{1+10^{-6}}}\approx\sqrt{2}\cdot 10^{-3}\), ordine del mrad.
        \end{flushright}
        <<Bene, quindi lei ha questo pione che è appena sopra la soglia, dei fotoni emessi a qualche mrad. Quanto dev'essere spesso questo gas? Dobbiamo assicurarci che il fotone venga prodotto dal materiale. Quanti fotoni vengono prodotti nel materiale?>>
        \begin{flushright}
        <<Empiricamente, abbiamo circa \(\frac{\textrm{d}N}{\textrm{d}x}\sim 500 \sin^2\theta_c=500\left(1-\frac{1}{\left(\beta n\right)^2}\right)\) fotoni/cm>>.
        \end{flushright}
        Ci vorranno decine di metri.
        }
        \item Chiara: Come funziona un fotomoltiplicatore (PMT) e per cosa lo usa?\\
        \textit{\begin{flushright}
        *inizia a disegnare un PMT* \\<<È diviso in due sezioni: una è la piastra del catodo e all'interno del PMT c'è una serie di dinodi.>><<...efficienza quantistica...>><<...Aumento l'intensità del fotoelettrone così vedo meglio il fotone incidente>>
        \end{flushright}
        Cavoto:<<Cosa intente con "aumento l'intensità del fotoelettrone"?>>
        \begin{flushright}
        <<La corrente>>
        \end{flushright}
        <<Ah, esatto. E come si aumenta questa corrente? Non basta che ci sia un voltaggio sempre più alto.>>
        \begin{flushright}
        *Disegna il circuito equivalente*
        \end{flushright}
        <<No, ma io le sto chiedendo una cosa più semplice. Il fotocatodo, cosa esce dal fotocatodo?>>
        \begin{flushright}
        <<I fotoelettroni>>
        \end{flushright}
        <<Quanti ne escono? Dipende dal fotone incidente e...? Di che energie parliamo del fotone incidente?>>
        \begin{flushright}
        <<Ordine dell'eV, essendo effetto fotoelettrico. E anche il fotoelettrone.>>
        \end{flushright}
        C:<<Benissimo, e quando esce fuori cosa trova? Fuori dal fotocatodo? I dinodi saranno messi in una scatola, e nella scatola oltre i dinodi cosa c'è? Anzi, cosa non c'è?>>
        \begin{flushright}
        <<Ci devo mettere un materiale non assorbente, come un gas>>
        \end{flushright}
        C: <<Se ci mette un gas, cosa succede a un elettrone da 1 eV? Dev'esserci il vuoto. Bene. Adesso manca un collegamento per generale l'aumento di corrente di cui lei mi parlava. L'elettrone con quale energia arriva sul primo dinodo?>>
        \begin{flushright}
        <<Dell'ordine dell'eV.>>
        \end{flushright}
        C: <<Sì? E allora questa storia delle differenze di potenziale a che serve?>>
        \begin{flushright}
        <<No, viene accelerato dal dinodo, quindi arriva con un'energia superiore>>
        \end{flushright}
        C: <<Se la differenza di potenziale tra il catodo e il dinodo è 100 V, con quanta energia cinetica arriva? 100 eV. E che succede sul dinodo?>>
        \begin{flushright}
        <<Si crea una valanga sfruttando la produzione di elettroni e lacune in un semiconduttore.>>
        \end{flushright}
        <<No... Sfrutta la proprietà chiamata "secondary emission".>>
        }\\
        \textbf{Voto:} 28 https://www.overleaf.com/project/5f6dc4cc7e251a0001995129
    \end{enumerate}
    \begin{itemize}
        \item Discutere di un materiale scintillante. Ho scelto lo Ioduro di Sodio. Di questo occorre saper stimare la lunghezza di radiazione (ordine del centimetro) e l'energia critica. Dato un fotone in ingresso da 100 Mev, bisogna stimare la lunghezza del calorimetro. Mi è stato utile, poi, ricordare la risoluzione tipica di una camera proporzionale a molti fili.
    \end{itemize}
    \begin{enumerate}
        \item Come misuro l’energia di un fotone ad 1 MeV e i rivelatori di fotoni.
        \item È stata la sua assistente a chiedere la parte sui PMT e affini.
    \end{enumerate}
\paragraph{22 Gennaio 2021}
    \begin{enumerate}
        \item Supponiamo di poter misurare il tempo di vita medio, o \textit{lifetime}, dei muoni che sono nei raggi cosmici. Se io volessi misurarla con un'incertezza del per cento, che tipo di sistema/rilevatore userei, che sia il più semplice possibile? E quanto tempo dovrei passare prima di avere una misura con questo livello di incertezza?\\
        \textit{
        \begin{flushright}
        *descrive i raggi cosmici*
        \end{flushright}
        <<Assumiamo che siamo al livello del mare, qual è il tasso, o \textit{rate}, dei raggi cosmici? E assumiamo che abbiano una quantità di moto pari a 1 GeV. Come farebbe? Con quale strumentazione? E come stima l'incertezza?>>
        \begin{flushright}
        <<Qui dobbiamo fare un calcolo di conteggio>>
        \end{flushright}
        }
        \texttt{perso.}
    \end{enumerate}
    \begin{enumerate}
        \item Contatore Čerenkov;
        \item TPC.
    \end{enumerate}
    \begin{enumerate}
        \item Come misurare il decadimento di un pione neutro.
        \item Camere a deriva ({\it Drift chambers}).
    \end{enumerate}
    \begin{enumerate}
        \item Come misurare efficienza di un fotone in un contenitore con un gas
        \item PMT\\
        \textit{cosa sono, gain, efficienza}
    \end{enumerate}
    \begin{enumerate}
        \item A me ha detto di immaginare di voler misurare l'attività di una sorgente composta da un radionuclide, il sodio 22, $^{22}$Na, aggiungendo che decade $\beta^+$. Il positrone sta dentro la sorgente stessa e possiede una certa energia. Cosa esce dalla sorgente secondo lei? Supponi la sorgente sia molto spessa, un blocchetto immerso in un materiale tale che il range del positrone rimanga dentro.
        \textit{
        \begin{flushright}
        S:\textcolor{Violet}{<<Considerando che è un decadimento a tre corpi: $^{22}Na \rightarrow X +e^++\nu_e$>>}
        \end{flushright}
        C:<<Cosa vuol dire che il range rimane dentro?>>
        \begin{flushright}
        \textcolor{Violet}{<<Il range è definito come \(R=\int_E^m\frac{\textrm{d}E}{\frac{\textrm{d}E}{\textrm{d}x}}\)>>}
        \end{flushright}
        C: <<Diciamo che il positrone perde energia solo per ionizzazione. Per essere sicuri di questa cosa, cosa dovrei andare a controllare?>>
        \begin{flushright}
        \textcolor{Violet}{<<Energia critica del materiale e l'energia della particella. Siccome voglio che il range della particella sia minore dello strato del materiale, possiamo anche scegliere un materiale pesante, però in generale la formula di Bethe e Block...>>}
        \end{flushright}
        <<Nono, il range stiamo tranquilli, a questo punto, se una particella arriva \textit{a range}, cosa vuol dire?>>
        \begin{flushright}
        \textcolor{Violet}{<<Che viene assorbita>>}
        \end{flushright}
        <<Che si ferma, a questo punto non esce niente dal blocchetto?>>
        \begin{flushright}
        \textcolor{Violet}{<<Si annichilerà e produrrà fotoni: \(e^+e^-\rightarrow\gamma\gamma\). Se usciranno o meno dipende dall'energia dei fotoni, considerando che il positrone è fermo al più faranno effetto fotoelettrico.>>}
        \end{flushright}
        <<Perché?>>
        \begin{flushright}
        \(\sqrt{EE_1E_2\left(1-\cos{\theta}\right))}=\)
        \end{flushright}
        <<Il centro di massa non è diverso dal laboratorio>>
        \begin{flushright}
        <<$\sqrt{S}=2m_e$. Questi due fotoni sono emessi nel cdm a 180$^\circ$ e anche nel laboratorio.>>
        \end{flushright}
        <<In quale caso cambierebbe nel laboratorio? Considerato che abbiamo assunto il positrone fermo a fine range, qual è l'altra variabile cinematica che resta?>>
        \begin{flushright}
        <<L'impulso dell'elettrone, dovrei considerare il $\beta$ dell'elettrone>>
        \end{flushright}
        <<E che $\beta$ avrebbe?>>
        \begin{flushright}
        <<$\beta=\frac{p_e}{E_e}$, al massimo hanno l'impulso di Fermi, circa 5 Mev. Dipende anche dai tempi di interazione.>>
        \end{flushright}
        <<Dove sta questo elettrone?>>
        \begin{flushright}
        <<Nel mezzo, è comunque legato e quindi avrà un'energia che è minore dell'energia di legame, che è dell'ordine dell'eV. \(\beta=\frac{p_e}{E_e}\approx\frac{eV}{0.511\cdot10^6\textrm{eV}}\).>>
        \end{flushright}
        <<Quella però è l'energia di legame, non il suo impulso>>
        \begin{flushright}
        <<$E=k+M$, $\beta$ è trascurabile>>
        \end{flushright}
        <<Va bene, questi due fotoni che energia hanno?>>
        \begin{flushright}
        <<Usando la conservazione del quadrimpulso e il fatto che vengono emessi a 180 gradi e $E_1=E_2$:\\ \(\sqrt{EE_1E_2\left(1-\cos{\theta}\right))}=2m_e\)\\
        \(\cancel{2}\sqrt{E_1E_2}=\cancel{2}m_e\)\\
        \(E=m_e\)
        \end{flushright}
        <<Supponiamo che lei voglia rivelare questi fotoni, come fa?>>
        \begin{flushright}
        L'ordine di energia è della massa dell'elettrone, quindi abbiamo effetto Compton (anche se non dominante) e effetto fotoelettrico. Quindi potrei usare un PMT, un fotomoltiplicatore, faccio vedere il principio di funzionamento?>>
        \end{flushright}
        <<È la scelta giusta?>>
        \begin{flushright}
        Il fotomoltiplicatore sfrutta l'effetto fotoelettrico, l'unico problema è che forse siamo in un regime di energia già troppo alto. Quindi conviene passare direttamente a un calorimetro o a un fotodiodo.>>
        \end{flushright}
        <<Calorimetro? Fotodiodo? Calorimetro o fotodiodo?>>
        \begin{flushright}
        <<Non siamo ancora nel range della produzione di coppia, quindi il fotodiodo>>
        \end{flushright}
        <<Il fotodiodo... E qual è la differenza tra fotomoltiplicatore e il fotodiodo?>>
        \begin{flushright}
        <<Entrambi sfruttano l'effetto foto elettrico, nel fotodiodo una volta che l'energia è maggiore dell'energia del gap, $E>E_g$ io ho effettivamente la produzione dell'elettrone e posso rivelare il fotone. Anche qui però siamo in regimi di energia troppo grandi.>>
        \end{flushright}
        <<Se lei ha un fotone da 500 keV, come in questo caso, per assorbirlo quanto materiale ci vuole?>>
        \begin{flushright}
        <<La lunghezza di assorbimento è per definizione $\lambda=\frac{1}{n\sigma}$, $\sigma_{\textrm{fotoelettrico}}= \frac{Z^5}{E_{\gamma}^3}$, $\sigma_{\textrm{c}}=\frac{Z}{E_\gamma}$. Il fotodiodo è a Silicio, quindi è predominante...>>
        \end{flushright}
        <<Quanto è grosso un fotodioco?>>
        \begin{flushright}
        <<Abbiamo come spessore il micron, forse millimetri>>
        \end{flushright}
        <<300 $\mu m$. Quello che voglio dire è che se lei ha un materiale spesso 300 $\mu m$, lei giustamente deve calcolare la lunghezza di assorbimento. Però questa lunghezza dipenderà anche dalla densità, per esempio. Ora in 300 $\mu m$, 500 keV non si assorbono. Per cui, cosa è meglio fare?>>
        \begin{flushright}
        <<Si potrebbe prendere un materiale scintillatore e farlo seguire da un fotodiodo di modo che, in parte venga assorbita parte dell'energia del fotone, così da arrivare in un range buono per il fotodiodo e venga assorbito.>>
        \end{flushright}
        Ass:<<Si può fare qualcosa meglio di così? Lei ha citato un materiale scintillante, qual è lo scopo del materiale scintillante?>>
        \begin{flushright}
        <<Misurare l'energia del fotone>>
        \end{flushright}
        <<E in che modo? Cosa fa?>>
        \begin{flushright}
        <<Produce elettroni>>
        \end{flushright}
        <<Perché si chiama scintillante?>>
        \begin{flushright}
        <<Produce fotoni>>
        \end{flushright}
        <<Di che energia? Che ordine di grandezza?>>
        \begin{flushright}
        <<Ordine dell'elettronvolt, per gli inorganici erano 3 eV che posso vedere con un PMT senza problemi con l'effetto fotoelettrico.>> *Descrive un PMT*
        \end{flushright}
        C:<<La risoluzione energetica di quest'oggetto alla fine da cosa dipende? In prima approssimazione.>>
        \begin{flushright}
        <<Dal numero di elettroni>>
        \end{flushright}
        <<Di fotoelettroni, quelli che escono dal fotocatodo $\frac{1}{\sqrt{N}}$>>
        }
    \end{enumerate}
    \textbf{Voto:} 27
    \begin{enumerate}
        \item Supponga di avere una sorgente $\beta$ che emette su tutto l'angolo solido, ma in maniera non uniforme. Definiamo tanto per cominciare un sistema di riferimento $xy$, la sorgente sta nell'origine e chiamiamo $\theta$ l'angolo rispetto all'origine e dobbiamo misurare la distribuzione angolare di queste particelle $\beta$. In realtà sappiamo che questa distribuzione angolare ha un andamento proporzionale a $\cos\theta$. Per cui lo spettro di questi $\beta$ è circa \(\frac{\textrm{d}N}{\textrm{d}\cos\theta}\sim \cos\theta\). Come costruirebbe un rivelatore capace di misurare questa distribuzione angolare? Intanto mi dica cosa vuol dire particella $\beta$.
        \textit{
        \begin{flushright}
        <<È un elettrone>>
        \end{flushright}
        <<E l'energia di questo elettrone se mi dovesse disegnare lo spettro dell'energia?>>
        \begin{flushright}
        \(X_A^{Z}\ \rightarrow \ X_Z^{Z+1}+e^-+\overline{\nu}_e\)\\ *Marca un punto sull'asse $\hat{x}$ come energia massima*
        \end{flushright}
        <<E questa energia massima da cosa dipende? Normalmente è chiamata \textit{Q-value}>>
        \begin{flushright}
        <<Energia del nucleo all'inizio meno l'energia dell'elettrone e del neutrino. Siamo nel caso in cui questi due corpi hanno energia cinetica nulla e quindi tutta l'energia cinetica la ha l'elettrone.\\
        \(n \ \rightarrow \ p+ e^-+\overline{\nu}_e\)\\
        \(q=E_n-m_p-m_{\overline{\nu}_e}\)>>
        \end{flushright}
        <<Diciamo pure che il nuclide che decade è fermo.
        \begin{flushright}
        *lo studente corregge $E_n$ con $m_n$ del neutrone*
        \end{flushright}
        Quello che lei ha scritto è vero nel decadimento del neutrone, ma in generale, cosa ci dovremmo mettere? Nel decadimento di X.>>
        \begin{flushright}
        <<La massa del nucleo>>
        \end{flushright}
        <<Che è uguale alla massa dei suoi costituenti?>>
        \begin{flushright}
        <<No, c'è anche l'energia di legame>>
        \end{flushright}
        <<Detto questo, mi disegna lo spettro?>>
        \begin{flushright}
        <<Non mi ricordo>>
        \end{flushright}
        <<È monocromatico?>>
        \begin{flushright}
        <<No, perché è un decadimento a tre corpi. E non sarà nemmeno costante>>\\
        *La disegna*
        \end{flushright}
        <<Benissimo, se noi volessimo misurare la distribuzione angolare di questi elettroni, come farebbe lei? Assuma la sorgente puntiforme>>
        \begin{flushright}
        <<Metterei la sorgente al centro, magari di una "drift chamber", di modo che posso misurare l'angolo di emissione>>
        \end{flushright}
        <<Eh, ma per misurare l'angolo con la "drift chamber" come fa?>>
        \begin{flushright}
        <<Vediamo come funzione, nella "drift chamber" c'è un campo elettrico e contiene un gas, l'elettrone lascia una traccia ionizzando il gas e rilasciando elettroni per ionizzazione che vengono mossi dal campo elettrico e arrivano all'anodo. L'anodo è segmentato, per esempio da una serie di quadratini. Noi assumiamo che gli elettroni vadano con una certa velocità verso l'anodo ammettendo una certa diffusione. Quindi abbiamo il nostro anodo che è diviso in tanti quadratini e se misuriamo l'arrivo degli elettroni nei quadratini possiamo misurare l'angolo>>
        \end{flushright}
        <<Questa cosa qua, che in linea di principio potrebbe anche essere possibile, ha un problema. Voglio dire, lei così misura l'angolo, però cosa succede a questi elettroni... Questi elettroni $\beta$... $Q$ che energia potrebbe avere? L'ordine di grandezza>>
        \begin{flushright}
        <<Qualche decina di MeV?>>
        \end{flushright}
        <<Anche meno, diciamo 1 MeV. Per fare quello che dice lei deve passare nel gas>>
        \begin{flushright}
        <<Però $\beta\gamma = 2$, potrebbe essere in regine di Bethe-Block.>>
        \end{flushright}
        <<Ok... A quale altro fenomeno va incontro attraversando il gas?>>
        \begin{flushright}
        <<diffusione coulombiana multipla: l'elettrone tende ad avere impatti che cambiano l'angolo, quindi questo non è un buon modo per...>>
        \end{flushright}
        <<Eh, vabbè, ma quant'è questo effetto?>>
        \begin{flushright}
        <<Allora, l'angolo medio della diffusione multipla è zero, ma la varianza di questo angolo \(\sigma_\theta=\theta_0\propto\sqrt{\frac{x}{X_0}}\).Più precisamente \(\sigma_\theta=\theta_0=\frac{136 MeV}{\beta c p}\sqrt{\frac{x}{X_0}}\)e $X_0$ la lunghezza di radiazione.
        \end{flushright}
        <<Quanto sarà la lunghezza di radiazione della camera a gas, più o meno, a pressione atmosferica>>
        \begin{flushright}
        <<\(\rho X_0\simeq 10 g/CM^2 \ \rightarrow \ X_0=10^4cm=100m\). $x$ diciamo che per semplicità è circa $1 m$, anche se è troppo grande>>
        \end{flushright}
        <<Lei mi ha detto per misurarlo uso questa camera a gas, che è tipo una TPC e uso la proiezione della traiettoria. Io le ho detto, sì però c'è anche la diffusione multipla, che lei sta calcolando, eh diciamo, "x" cosa conviene mettere?>>
        \begin{flushright}
        <<Il raggio della camera>>
        \end{flushright}
        <<Se la camera fosse un chilometro? Questo coso, come dire, $\sigma_\theta$ come diventa?>>
        \begin{flushright}
        <<Diciamo che l'elettrone esce radiale e che il raggio sia un metro...>>
        \end{flushright}
        <<Quindi lei vorrebbe stimare $x$ come un metro però... Diciamo che tutte le cose hanno delle complicazioni e quindi dei costi, quindi la domanda potrebbe essere, quand'è che vale la pena e non spendere più soldi? Un chilometro è tanto, un metro è fattibile, ma vale la pena con i conti che ha fatto? Se lei mette un metro nella sua formula cosa ottiene?>>
        \begin{flushright}
        \(\beta=\frac{p}{E}=\frac{\sqrt{E^2-m^2}}{E}=\sqrt{1-\frac{m^2}{E^2}}\simeq 1+\frac{1}{2}\frac{m^2}{E^2}\)
        \end{flushright}
        <<Ma è autorizzato a fare questa espansione in serie?>>
        \begin{flushright}
        <<No, perché...>>
        \end{flushright}
        <<"m" cos'è? E "E" quant'è invece?>>
        \begin{flushright}
        <<"m" è la massa dell'elettrone, E=1 MeV>>
        \end{flushright}
        <<Sicuro che E=1 MeV?>>
        \begin{flushright}
        <<No, 1 MeV è l'energia cinetica massima, quindi se prendiamo il massimo allora l'energia dell'elettrone è circa 1,5 MeV\\
        $\beta=\sqrt{\frac{8}{9}}=\frac{2\sqrt{2}}{3}\approx 1$\\
        $p=\sqrt{E^2-m^2}\approx\sqrt{2} MeV$>>
        \end{flushright}
        <<Quindi se mette il numero là dentro, cosa le esce fuori?>>
        \begin{flushright}
        <<\(\theta_0=\frac{10}{\frac{\sqrt{2}}{3}\sqrt{2}}\sqrt{\frac{1}{100}}=\frac{3}{8}\) rad. Che è enorme>>
        \end{flushright}
        <<Esatto, è enorme, come si può migliorare questa cosa? Come si può cambiare il rivelatore per evitare il problema della diffusione multipla?>>
        \begin{flushright}
        <<Possiamo cambiare solo la dimensione della camera e la pressione del gas contenuto>>
        \end{flushright}
        <<C'è un modo per cambiare il gas, oltre al cambiare il tipo?>>
        \begin{flushright}
        <<Se diminuiamo la pressione, diminuiamo la densità>>
        \end{flushright}
        <<Ok, va bene. Visto che siamo qua, una cosa simile a quella che lei ha proposto è una camera a fili, mi sa dire rispetto alla TPC come funziona la camera a fili? Intanto disegni una camera a fili tutta>>
        \begin{flushright}
        *disegna la camera a fili*\\
        <<Il funzionamento è simile, se abbiamo un elettrone che passa, la particella lascia una scia di elettroni e positroni. Da quanta carica arriva sul filo possiamo stimare l'energia dell'elettrone, perché elettroni a energie più alte generano più portatori di carica>>
        \end{flushright}
        <<È vera questa cosa? Questi elettroni come sono generati? Secondo quale legge?>>
        \begin{flushright}
        <<Ecco, non necessariamente sono proporzionali all'energia perché c'è la formula di Bethe-Block>>\\
        *disegna il grafico*
        <<Cioè il numero di elettroni è proporzionale alla perdita di energia, la perdita di energia dipende dall'energia dell'elettrone, però in maniera non lineare.>>
        \end{flushright}
        <<Lei dice vengono raccolti sul filo questi elettroni primari generati nel gas, questo è quello che genera il segnale che vediamo?>>
        \begin{flushright}
        <<No, perché c'è un altro effetto per cui gli elettroni mentre si avvicinano al filo accelerano e nell'accelerare possono generare una valanga, cioè possono ionizzare altri atomi o eventualmente radiazione che può ionizzare altri atomi. Quindi sul filo ci sono molti più elettroni>>
        \end{flushright}
        <<E perché accelerano?>>
        \begin{flushright}
        <<Perché il campo elettrico avvicinandosi al filo aumenta>>
        \end{flushright}
        <<Ok. In questo caso ci sono lo stesso fenomeni di diffusione, no? Se lei dovesse poi valutare la traiettoria dell'elettrone, quindi anche l'angolo di cui parlavamo nell'esercizio precedente, quali informazioni potrebbe ricavare da un rivelatore fatto così>>
        \begin{flushright}
        <<La cosa più semplice se ho tante camere di questo tipo è di considerare questa come un pixel, ma questa non è molto efficiente.>>
        \end{flushright}
        <<Diciamo queste celle cosa possono misurare oltre alla carica raccolta?>>
        \begin{flushright}
        <<Da sole possono misurare solo questo, ma se aggiungo un altro rivelatore che mi dice quando la mia particella entra e esce dalla camera a fili, possiamo stimare il tempo che ci mettono gli elettroni ad arrivare al filo e sapendo la velocità di deriva nel gas possiamo avere una stima della distanza>>
        \end{flushright}
        <<Mi dice che risoluzione tipica ha su questa distanza?>>
        \begin{flushright}
        <<Ci sono due effetti che contribuiscono sulla risoluzione, il fatto che se la traiettoria è molto vicina gli elettroni potrebbero non generarsi sulla perpendicolare tra filo e traiettoria e quindi metterci più tempo>>
        \end{flushright}
        }
        \texttt{perso.}
    \end{enumerate}
    \begin{enumerate}
        \item Supponga di avere un esperimento in cui ci sono particelle cariche stabili, per cui ci sono elettroni, muoni, pioni, kaoni e protoni e poi di avere un rivelatore a luce Čerenkov, fatto di silicio fuso ("\textit{fused silica}"), che dovrebbe sapere cos'è: è un materiale trasparente che ha un indice di rifrazione di 1,5. Mi sa disegnare per queste 5 specie la relazione che c'è tra l'angolo Čerenkov e l'impulso? L'intervallo dell'impulso di queste particelle è tra 100 MeV e 3 GeV e ognuna ha impulso diverso. Vorrei che mi disegnasse 5 curve mettendo sull'asse delle $x$ l'impulso e sull'asse delle $y$ l'angolo Čerenkov. 
        \textit{
        \begin{flushright}
        <<Intanto vediamo che per l'effetto Čerenkov è un effetto di soglia, che è quando \(\beta>\frac{1}{n}\) e l'angolo di emissione è uguale a $\cos\theta_c=\frac{1}{\beta n}$. Potrebbe essere utile calcolare i vari angoli delle particelle.>>
        \end{flushright}
        <<Lei saprebbe descrivere da dove ha tirato fuori questa formula?>>
        \begin{flushright}
        <<Consideriamo una particelle che si muove lungo l'asse $z$ con velocità $v=\beta c$, emette radiazione in maniera uniforme intorno a se e spostandosi si sposta anche il fronte d'onda.\\
        *disegna il fronte d'onda*\\
        \(R=\Delta t c\) mentre la distanza $d=c\beta\Delta t \cos{\theta_c}=\Delta t c_n \ \rightarrow \cos\theta_c=\frac{1}{\beta n}$
        \end{flushright}
        <<Va bene, se questa è la relazione, essendo "n" fissato, cosa mi disegna?>>
        \begin{flushright}
        <<Io intanto so che $\beta=\frac{p}{E}=\frac{p}{\sqrt{P^2+m^2}}$. Se io aumento l'impulso $\beta$ tende a 1 e il $\cos{\theta_c}$ a $\cos\theta_c=\frac{2}{3}$. Quello che mi aspetto la funzione dell'impulso tenda a questa soglia tanto più velocemente quanto è...>>
        \end{flushright}
        <<Perché la soglia?>>
        \begin{flushright}
        <<Intendo un asintoto, un limite scusi. Tendono a questo asintoto con velocità diversa che dipende dalla massa>>
        \end{flushright}
        <<La domanda è, cos'è che cambia in quelle curve. Da dove partono?>>
        \begin{flushright}
        *Nell'immagine del ragazzo a $p=0 \ \rightarrow \theta=0$*
        \end{flushright}
        <<Se un protone ha impulso 100 Mev, quant'è il suo $\theta_c$?>>
        \begin{flushright}
        <<Il protone ha una massa di 500 MeV...>>
        \end{flushright}
        <<Qual è la massa del protone?>>
        \begin{flushright}
        <<900 MeV, quasi 1 GeV. Se ho $\frac{100}{\sqrt{10^4+10^6}}=\frac{10^2}{10^3}=0.1 \ \rightarrow \ \cos\theta_c=\frac{1}{0.1\cdot 1,5}=\frac{100}{15}=\frac{20}{3}$. Il coseno è maggiore di 1 quindi non c'è emissione Čerenkov. Queste curve dovrebbero quindi partire dalla soglia.>>
        \end{flushright}
        <<Quand'è di poco sopra la soglia, l'angolo Čemetodirenkov quant'è?>>
        \begin{flushright}
        <<è 0>>\\
        *cambia il disegno e a $\theta=0$ non fa più iniziare le curve a $p=0$*
        \end{flushright}
        <<Quindi per finire questo esercizio, lei dovrebbe dirmi... visto che in questa scala partiamo da 100 MeV... Dovrebbe dirmi cosa corrisponde a cosa. Come associa le 5 particelle alle 5 curve?>>
        \begin{flushright}
        <<Mi aspetto che la prima curva sia quella dell'elettrone e poi a salire con le masse. Elettrone, muone, pione, kaone>>
        \end{flushright}
        }
        \item Raccontami come funziona un fotodiodo.
        \textit{
        \begin{flushright}
        <<Un fotodiodo sfrutta il meccanismo delle giunzioni PN per i     semiconduttori. Prendo due materiali con un certo drogaggio (aggiungo o sottraggo al materiale un elemento che abbia carenza o abbondanza di cariche positive o negative). Quando avvicino le due parti della giunzione PN si crea una differenza di potenziale tra la zone con le buche e quella con gli elettroni. Se una carica passasse all'interno della giunzione subirebbe un'accelerazione dovuta al campo elettrico all'interno. Quindi il meccanismo è...>>
        \end{flushright}
        <<Detto in maniera più rozza, P ed N sono due materiali carichi? Sono materiali che hanno uno un'abbondanza di cariche positive e uno di cariche negative?>>
        \begin{flushright}
        <<No, non sono...>>
        \end{flushright}
        <<Cosa si indica con P e N in un diodo normale, che uno può usare in fotodiodo oppure no? Detto rozzamente è un materiale in cui ci sono cariche libere di muoversi, non un'abbondanza di...>>
        \begin{flushright}
        <<Certo, certo. Quello che succede è che le cariche negative tengono ad occupare le zone positive del materiale di tipo P. In questo passaggio si crea una zona di equilibrio in cui lo spostamento delle cariche negative viene controbilanciato dalla creazione di un campo elettrico. Equilibrio che viene rotto dall'arrivo di un fotone sul fotodiodo che, tramite processi di scattering, libera degli elettroni che possono attraversare la zona intermedia del semiconduttore e arrivare a una zona di rivelazione posta nella parte opposta>>\\
        *disegna lo schema*
        <<Il motivo per cui si usano fotodiodi a semiconduttore è sia perché sono in grado di rivelare fotoni di energia nettamente superiore ai fotomoltiplicatore, sia perché l'efficienza è più alta, 80\% contro il 20/30\%>>
        \end{flushright}
        <<Senta, come fa a vedere un fotone de 1 KeV con un fotodiodo. Lei ha detto che può vedere fotoni di energia superiori a quelli di un fotomoltiplicatore, che forse è anche vero, ma cosa succede a un fotone da 1 KeV in questo fotodiodo?>>
        \begin{flushright}
        <<Quello che succede è che si verifica il fenomeno dello sciame elettromagnetico, o "shower" elettromagnetica e il fotone...>>
        \end{flushright}
        <<Un fotone da 1 KeV può fare uno sciame?>>
        \begin{flushright}
        <<Nono, intendevo un effetto fotoelettrico. L'elettrone emesso per effetto fotoelettrico viene accelerato dal campo della giunzione e attraversa il semiconduttore e può essere rivelato alla fine.>>
        \end{flushright}
        <<Un elettrone solo?>>
        \begin{flushright}
        <<No, non è che un fotone produce un solo fotoelettrone>>
        \end{flushright}
        <<Un fotone da 1 Kev, quanti elettroni produce?>>
        \begin{flushright}
        <<Il numero di fotoni prodotti in funzione della distanza percorsa è circa \(\frac{\textrm{d}N}{\textrm{d}x}\approx \frac{500}{-}\)
        \end{flushright}
        <<Questo è il numero di fotoni Čerenkov. Che energia possiede il fotoelettrone creato dal fotone di 1 Kev?>>
        \begin{flushright}
        <<Per effetto fotoelettrico ne genera solo uno con energia, con energia dell'ordine...>>
        \end{flushright}
        <<No, è un numero abbastanza preciso, è 1 KeV, a parte l'energia di legame. Poi questo elettrone da un KeV ionizzerà e vabbè.>>
        }
    \end{enumerate}
    \textbf{Voto:} 25
    \begin{enumerate}
        \item Supponiamo di voler misurare la divergenza (quindi la distribuzione degli angoli) di un fascio di muoni che hanno quantità di moto 100 GeV e che viaggiano nel vuoto in un tubo. Ogni muone ha una direzione diversa rispetto all'asse orizzontale. Quindi per ogni particella bisogna misurare il suo angolo, come farebbe lei?>>
    \textit{\begin{flushright}
    <<Se viaggiano nel vuoto, non andranno incontro a diffusione coulombiana multipla. Io utilizzerei due scintillatori orientati in direzione ortogonali al fascio. In questo modo i muoni andranno a eccitare.. Allora no, metterei un ammontare di "wire chambers" in modo tale da avere all'interno di ogni "chamber" la produzione di elettroni in modo che arrivino all'anodo e dopo di che, dopo che ogni anodo viene mappato nella mappa bidimesionale che mi sono fatto posso risalire all'angolo di passaggio del muone>>\\
    *Disegna una circonferenza con dei puntini perpendicolare al fascio*
    \end{flushright}
    <<Questo rivelatore che ci sta disegnando... Questi punto cosa sono?>>
    \begin{flushright}
    <<Questi punti sono anodi e appartengono ognuno a una differente "wire chamber". Sono i fili anodici uscenti dal foglio>>
    \end{flushright}
    <<Cosa misurano questi fili? Cioè lei deve misurare delle posizioni no?>>
    \begin{flushright}
    <<Questi fili mi misurano il passaggio o meno del muone grazie al fatto che in essi possono arrivare, o meno, elettroni liberati dal passaggio del muone>>
    \end{flushright}
    <<Qual è la risoluzione angolare di questo sistema?>>
    \begin{flushright}
    <<Ipotizziamo di aver posizionato le nostre "wire chambers"... Esse hanno uno spessore di 1 cm...>>
    \\
    *Intorno ad ogni punto che ha disegnato nella circonferenza grande, disegna piccoli cerchi*
    \end{flushright}
    <<Cosa intende per diverse "wire chambers"? Questi altri cerchi che ha disegnato cosa sono?>>
    \begin{flushright}
    <<I punti sono differenti fili anodici, ognuno all'interno della propria camera, che sono schematizzate da circonferenze piccole.>>
    \end{flushright}
    <<Quando lei parla di un anodo, è un oggetto posto a un potenziale positivo o negativo?>>
    \begin{flushright}
    <<Positivo>>
    \end{flushright}
    <<Quindi lei ha tanti oggetti posti a potenziale positivo,  quindi un elettrone che passa in mezzo a due oggetti posti a potenziale positivo...>>
    \begin{flushright}
    <<No, perché ogni cerchio è una singola camera anodica>>
    \end{flushright}
    <<Eh, ma perché l'elettrone vuole andare su quel filo [uno dei puntini nel cerchio] e non su un altro?>>
    \begin{flushright}
    <<Perché è più vicino>>
    \end{flushright}
    <<Il campo lì dentro com'è?>>
    \begin{flushright}
    <<Il campo è proporzionale a $\frac{1}{r}$>>
    \end{flushright}
    <<Scusi, se lei mette tante cariche positive vicine, equidistanti, il campo elettrico tra due punti...>>
    \begin{flushright}
    <<Sisi, sicuramente ci sarà il contributo anche dei campi elettrici degli altri fili anodici, però quello maggiore rimane quello del filo più vicino>>
    \end{flushright}
    <<Usciamo un attimo da questo schema, lei prende due cariche positive a una distanza "d", qual è il campo in un punto tra la linea che le congiunge? Per avere un campo elettrico per un anodo, lei ha bisogno di un riferimento, di un qualcosa, un riferimento di potenziale. Lei ha tutti punti allo stesso potenziale, l'intera superficie è equipotenziale.>>
    \begin{flushright}
    <<I fili anodici hanno lo stesso potenziale, però ad esempio, ciò che va a delimitare la "wire chamber" può essere messa a terra, in modo tale da non avere una superficie equipotenziale, ma un insieme discreto di punti che abbiano sì lo stesso potenziale. È un modo per ottenere una "time projection chamber", però con i fili anodici, è più economico perché per coprire il singolo $cm^2$ si ha bisogno di meno fini anodici rispetto ai pixel di una TPC.>>
    \end{flushright}
    <<In realtà no, questa cosa si fa mettendo un filo anodico circondato da qualche filo chiamato "fili di massa" in modo che generino un campo elettrico intorno al filo anodico, può essere o un tubo pieno, quasi come questi che ha generato lei, oppure altri fili che vengono messi a massa in modo che ci sia un campo elettrico vero. Altrimenti tutti i fili che lei ha messo sono tutti allo stesso potenziale, quindi l'elettrone non si muove da nessuna parte. Supponiamo che questi cerchi piccoli siano dei potenziali messi a massa, come fa a misurare l'angolo?>>
    \texttt{Secondo me era esattamente quello che ha detto il ragazzo, ma non lo avevano capito}
    \begin{flushright}
    <<In questo caso avrei una traiettoria di fili anodici che hanno ricevuto elettroni e che segnano il passaggio della carica e quindi l'angolo è data dalla posizione del filo anodico>>
    \end{flushright}
    <<E che tipo di segnale vedono questi fili anodici?>>
    \begin{flushright}
    <<Quando un filo anodico riceve gli elettroni...>>
    \end{flushright}
    <<Secondo lei quanti sono questi elettroni prodotti dentro...? Allora, innanzitutto di che dimensioni stiamo parlando, qual è la dimensione ragionevole di questi tubicini secondo lei?>>
    \begin{flushright}
    <<Un diametro ragionevole, penso sia del centimetro>>
    \end{flushright}
    <<Ok, e dentro c'è gas, secondo lei quanti elettroni vengono prodotti in un centimetro di gas?>>
    \begin{flushright}
    <<Pochi>>
    \end{flushright}
    <<Li calcoli, questo è un gas a pressione atmosferica>>
    \begin{flushright}
    <<Con Bethe-Bloch ho l'energia che deposita il muone>>
    \end{flushright}
    <<Intanto diciamo, Bethe e Bloch erano due che hanno scritto una formula, no?... Questa formula di cosa parla? Cioè, non è la reazione di Bethe e Bloch>>
    \begin{flushright}
    <<No questa formula parla del processo in cui, in modo quantitativo, vanno a perdere energia per ionizzazione del mezzo. Quindi appunto, va normalizzata per la densità del gas, che è collegata alla pressione del gas, abbiamo detto che siamo a pressione atmosferica quindi $\rho_{\textrm{aria}}=1,2\cdot10^{-3}g/cm^3$. Ora, in base alla perdita di energia... Sappiamo quanta energia è stata depositata all'interno della "wire chamber", se la dividiamo per l'energia che lega il singolo elettrone al nucleo atomico, si ottiene con approssimazione il numero di elettroni che ionizzati vanno fino...>>
    \end{flushright}
    <<Eh no, purtroppo... Questo sarebbe molto bello, ma siccome molti processi sono inefficaci, per avere il numero di elettroni prodotti, non si può semplicemente usare l'energia che lega un elettrone all'atomo, ma si utilizza se vuole anche dei valori quasi costanti, che però sono dei valori efficaci che tengono conto di tutti i processi, che sono di più, molti di più. Passiamo a un'altra domanda, sempre su questo.>>
    }
    \item Se poi vengono prodotti tanti elettroni, tanta carica, poi questo induce un segnali sui fili. Lei mi saprebbe dire come si calcola il segnale elettrico indotto su un elettrodo da una carica in moto?
    \begin{flushright}
    <<Se una carica in moto, per induzione... Con il teorema di Ramo>>
    \end{flushright}
    <<Sì, proviamo>>
    \begin{flushright}
    <<Non ricordo>>
    \end{flushright}
    <<Magari vuole pensare a tornare la prossima volta? Possiamo provare altri 5 minuti, ma arriviamo massimo a un 24/25.>>
    \end{enumerate}
\paragraph{27 Gennaio 2021}
    \begin{enumerate}
        \item A me ha detto di immaginare di voler misurare l'attività di una sorgente di sodio 22, aggiungendo che decade $\beta^+$.
        \item l'assistente (tale Cecilia) mi ha chiesto come distinguere pioni e kaoni con impulso a 1 GeV, ovviamente voleva sapere materiali e dimensioni di tutto.
        \item Ah chiede sempre quella storia di come evitare che i fotoni Compton scappino dal rilevatore coi conti.
    \end{enumerate}
    \begin{enumerate}
        \item A me ha chiesto come ricavare sperimentalmente il tempo di decadimento del muone;\\
        \textit{Che tipo di rilevatori useresti , come li posizioneresti ... Le domande non sono molto di tipo teorico}
    \end{enumerate}
\paragraph{18 Febbraio 2021}
    \begin{enumerate}
        \item Incertezza su angoli per disposizione a telescopio \item Multiple coulombian scattering (Molière).
        \item Cherenkov per distinguere particelle in fascio monocromatico.
    \end{enumerate}
    \begin{enumerate}
        \item A me ha chiesto TPC.\\
        \textit{principi di funzionamento.}
        \item Come rivelare la direzione di un fotone di 100 MeV.
    \end{enumerate}
\paragraph{24 gennaio 2022 - I appello}
\begin{itemize}
\item 
\begin{enumerate}
    \item Immagina di avere un fascio di adroni (pioni, kaoni, protoni) con impulso a 50 GeV, come costruiresti un sistema per distinguerli?\\
Ho scritto le masse di ciascuna particella e la mia idea era quella di usare tre rivelatori Cherenkov; ho calcolato l'indice di rifrazione minimo per aver luce Cherenkov (mi ha fermato dopo aver calcolato quello del pione), il punto era che usando gas l'angolo di emissione era molto piccolo (ho scritto la formula) e i fotoni molto pochi (ho scritto la formula), non ho capito come risolvere questo problema, ho citato di mettere degli specchi ma non so se non gli andava bene o non ha sentito o non l'ho sottolineato abbastanza)
\item Assistente [Bauce]: immagina di voler ricostruire posizione (e quindi l'angolo) con cui arrivano dei muoni nei raggi cosmici, che sistema useresti?\\ 
Io ho detto degli scintillatori con dei PMT alla fine, come in una diapositiva mostrata a lezione, mi ha chiesto la velocità effettiva della luce nel mezzo (circa la metà della velocità della luce nel vuoto: 15 cm/ns), quanto erano grandi questi scintillatori (ho detto 10 cm, ma non so se gli andava bene, non penso) e la stima dell'errore sulla posizione (ho assunto trascurabile l'incertezza sulla velocità effettiva e 1 ns quella sul tempo, il risultato era di circa 11 cm). L'assistente mi ha chiesto se volessi davvero conoscere l'errore sulla velocità effettiva cosa farei (l'assistente ha detto avrebbe fatto una calibrazione prima, io avevo capito dovessi usare sempre i muoni dei raggi cosmici).
\end{enumerate}
Voto: {\bf 30}
\item \begin{enumerate}
\item Prima domanda persa, ho visto che usava la configurazione a telescopio
\item  l'assistente [Bauce]: sorgente cobalto 60 $^{60}$Co, cos'è la vita media e come la misurerebbe. La ragazza dice di fare un istogramma contando gli elettroni, l'assistente chiede come mi aspetto che sia distribuito nel tempo (Cavoto interviene e dice: "in altre parole, cos'è N(t)?")
Cavoto chiede <<qual è il $\tau$ del $^{60}$Co? Il numero tipico.... Bé non c'è un numero tipico, se lo ricorda? Se fosse di un anno o tre giorni (quindi grande) un grafico più realista quale sarebbe?>> 
La ragazza dice mi aspetto una cosa lineare per intervalli di tempo minori (corretto), quindi espande l'esponenziale: \(N(t) = N_0\frac{t}{\tau}\).
Adesso parlando dei rivelatori: la ragazza dice di mettere scintillatori intorno per ottenere la misura della rate.
Assistente: cos'è questa rate?
La ragazza risponde
Assistente: quindi di questi conteggi cosa ottengo?
R = N/T ma dovrei tenere conto anche dell'accettanza geometrica... Finisce orale
\end{enumerate}
\end{itemize}
Voto: {\bf 30}
\paragraph{27 gennaio 2022 - I appello}
\begin{itemize}
\item \begin{enumerate}
    \item Come riuscire a distinguere muoni positivi da muoni negativi entrambi con un energia di 10 GeV?\\
    \textit{Gli ho detto di usare un campo magnetico e distinguerle in base alla curvatura. Mi ha fatto calcolare l'angolo di uscita delle due particelle (avevo messo $B=10\;T$ e un rivelatore quadrato di lato $L=1\;m$, facendo i conti sono passato a $L=10\;cm$), poi mi ha chiesto come faccio a rivelare il passaggio delle due particelle in uscita dal rivelatore? Metto due scintillatori di ioduro di sodio. A che distanza tra loro? Quanto devono essere spessi? Assumendo che per muoni a $10\;GeV$ valga $\frac{1}{\rho}\frac{dE}{dx}=2\;MeV/g/cm^2$, in $x=1\;cm$ vengono prodotti $\langle N \rangle=\frac{1}{\rho}\frac{dE}{dx}\rho x/W\sim10^5$ fotoni. Sono abbastanza? Breve discussione su fotomoltiplicatori e QE ma già mi ero perso.}
    \item Camere a deriva\\
    \textit{Gli ho fatto lo schema della camera a deriva mostrato a lezione, con lo scintillatore subito dopo la camera. Ho calcolato l'incertezza su $d$ assumendo nulla l'incertezza su $v_d$ e $\sigma_t\sim1\;ns$. Mi ha chiesto come poter ricostruire una traiettoria curvilinea con questo metodo e non gli ho saputo rispondere. (Da come ha detto rapidamente lui mettendone tante piccole vicine ma mi aveva già detto il voto quindi non l'ho ascoltato troppo).}
\end{enumerate}
Voto: {\bf 27}
\item \begin{enumerate}
    \item Cosa accade ad un fascio di muoni di energia di 1 GeV che vanno ad incontrare acqua al livello del mare?\\
    \textit{Discussione sulla perdita di energia secondo Bethe-Bloch e sul decadimento dei muoni. In che modo posso osservare quindi il passaggio delle particelle in acqua? Con l'effetto Cherenkov.}
    \item Fotomoltiplicatore\\
    \textit{Come funziona, com'è fatto, il guadagno (\textit{gain}), efficienza quantica.}
\end{enumerate}
Voto: {\bf 29}
\item \begin{enumerate}
    \item Come posso rivelare un fotone con energia di 100 keV?\\
    \textit{Discussione sulla sezione d'urto del fotone, cosa farà questo fotone? Effetto fotoelettrico, come posso misurarlo? Usando un fotodiodo.}
    \item Come funziona un fotodiodo?\\
    \textit{Com'è fatto, principio di funzionamento, gain.}
\end{enumerate}
Voto: {\bf 27}
\end{itemize}
\paragraph{17 febbraio 2022 - II appello}
\begin{itemize}
\item \begin{enumerate}
    \item \textbf{Esperienze}: E' molto tranquillo, ti dà il tempo necessario per farti pensare, considerando anche che alcune domande potrebbero essere un po' meno immediate di altre. Consiglio di preparare così laboratorio: iniziare con la teoria, concentrandosi sulle configurazioni sperimentali soprattutto (non chiederà mai invece dimostrazioni tipo quella di Bethe e Bloch). E' inoltre molto buono sapere a memoria alcune proprietà dei materiali (densità, lunghezze di radiazione, indici di rifrazioni, etc). Fatta la teoria, consiglio di fare gli esercizi svolti da lui in classe, che tende a chiedere, e sono un'ottima base da cui partire.
    \textbf{Orale}
Ha una sorgente alpha puntiforme con distribuzione ignota, come misura l'attività? Gli ho proposto una collezione di scintillatori attorno alla sorgente, affermando che vi doveva essere il vuoto all'interno. Che succede se P=10mbar? Cambia la densità, quindi DE (da fare il calcolo).
Assistente: Tubo Proporzionale (Gas Detector), come funziona? Discussioni varie sulle dimensioni. Come posso misurare il gain? Con un fit su Q=eGNi.
\textbf{Voto:} 30
    \end{enumerate}
\end{itemize}
\paragraph{21 febbraio 2022 - II appello}
\begin{itemize}
\item \begin{enumerate}
    \item Dato un fascio di pioni e muoni da 200 MeV, quanti ne ho dopo 20 m?\\
   \item Misurare direzione ed energia di un fotone da 50 MeV.
\end{enumerate}
Voto: {\bf 29}
\item \begin{enumerate}
    \item Stimare l efficenza di rilevare fotoni a 100kev in rilevatore gas\\
   \item p 1 GeV pioni e kaoni vanno su un quartzo di 3cm, cosa succede?
\end{enumerate}
\item \begin{enumerate}
    \item p muoni 1Gev, sigma 10microrad, vogliamo stimare l angolo\\
   \item avalanche e il gas detector
\end{enumerate}
\item \begin{enumerate}
    \item pioni e muoni 500mev, vuoi distinguerli con un rilevatore a gas\\
   \item fotone da 3Mev come lo rilevi?
\end{enumerate}
\item \begin{enumerate}
    \item energia cinetica pione 1Mev, quanta grafite e necessaria per fermarlo?\\
   \item vita media del pione come si misura?
\end{enumerate}
\item \begin{enumerate}
    \item muoni GeV quanti ne passano dopo 1km di acqua?\\
   \item come distinguere muoni positivi da quelli negativi?
\end{enumerate}
\item \begin{enumerate}
    \item z=10÷20 m=20GeV E=90GeV Come li rileviamo?(con cerenkov)\\
   \item come misurare l impulso di scontro di due particelle che producono particelle cariche?
\end{enumerate}
\item \begin{enumerate}
    \item sorgente radioattiva come misuri attivita?\\
   \item come rilevi muoni kaoni e elettroni ma non con cherenkov?
\end{enumerate}
\item \begin{enumerate}
    \item Dato un fascio di elettroni e muoni con p = 20 MeV, come costruire un rivelatore per distinguerli? 
    \item Come si calcola la risoluzione energetica in un calorimetro elettromagnetico?
\end{enumerate}

\end{itemize}

\paragraph{21 giugno 2022}
\begin{itemize}
    \begin{enumerate}
        \item Ha esordito chiedendomi come distinguo un fascio di elettroni e muoni di impulso fissato (20 Mev). L'esame è molto "interattivo" nel senso che lui ti segue in quasi tutto quello che dici (a meno che non siano cose proprio sbagliate) e ti fa varie domande su quello.\\
        Gli ho detto di poter sfruttare l'effetto Cherenkov (inizialmente pensavo per distinguere l'angolo di emissione) ed era ok, poi ho calcolato il $\beta$ per le due particelle e quello del muone era abbastanza piccolo mentre quello dell'elettrone praticamente 1. Quindi usciva fuori che per avere Cherenkov sul muone $n$ dovesse essere tipo 5, e mi ha fatto capire che è molto difficile, ma non è un problema visto che usando un materiale con $n$ tipici (intorno a 1,5) gli elettroni fanno Cherenkov e i muoni no quindi posso distinguerli. Mi ha chiesto che materiale potessi usare, ho detto un blocchetto d'acqua ($ n \sim 1,5$) in modo da avere un angolo di emissione ragionevole (all'inizio avevo detto un gas, ma non andava bene perchè è molto piccolo e quindi pochi fotoni). Mi ha chiesto quanto deve essere spesso questo blocchetto, è utile ricordarsi la formula molto rozza $\frac{dN}{dx} \sim 500sin^2(\theta_c)$ (e ricordarsi che è inteso come fotoni/cm). Gli ho detto di prendere un blocco di 1 metro circa (a caso) e mi ha fatto capire che non va bene, chiedendomi che altri effetti ci sono se un elettrone passa in un mezzo come l'acqua, gli ho detto che perde energia per ionizzazione e quindi per valutare se 1 metro va bene ho calcolato il range, mi sono messo a trovare il dE/dx con Bethe Bloch ma dopo un po' mi ha stoppato dicendomi che andava bene metterci nel MIP giusto per avere una stima approssimativa. Veniva che il range è dell'ordine di una decina di cm, quindi il blocchetto doveva essere dell'ordine dei cm e avrò circa 500 fotoni. \\
        Mi ha chiesto come rilevo questi fotoni, gli ho detto con un PMT e andava bene, mi ha chiesto dove va il fotone (fotocatodo) ma non non mi ha fatto descrivere il resto, invece mi ha chiesto la Q.E. (circa 20 per cento, quindi mi rimangono 100 fotoni in media) e poi mi ha chiesto a livello pratico come faccio a mandare questi fotoni nel PMT e quanti riesco a prenderne. All'inizio ho detto qualche tipo di guida ottica ma era sbagliato, mi ha fatto ragionare sul fatto che l'angolo Cherenkov sia una direzione ben fissata ma isotropa nell'angolo azimutale (un cono) quindi voleva sentirsi dire che di quei 100 fotoni prendo una percentuale che è pari all'angolo che copre il PMT rispetto a questo cono.
        \item L'assistente mi ha chiesto come misurare il tempo di decadimento dei muoni. Voleva intanto una stima dell'ordine di grandezza (2,2 $\mu s$) e poi quali rilevatori usare e in che modo. Sono partito dalla formula di decadimento esponenziale, prima nel CDM e poi nel Lab, $N(t)=N_0 exp(-\frac{t'}{\tau}) =N_0 exp(-\frac{t}{\gamma \tau})$ per poi passare all'espressione in funzione della posizione, $x = \beta c t$ quindi $N(x)=N_0 exp(-\frac{x}{\beta \gamma c \tau})$-\\
        Mi hanno chiesto di fissare un'energia, ho detto 10Gev mi pare (o 1 Gev non ricordo) e mi hanno fatto capire che era un po' tanto ma siamo andati avanti, allora mi hanno chiesto che rilevatori uso e come li metto e mi sono impicciato un sacco perché non capivo bene cosa volevano, sono stato zitto per qualche minuto mentre Cavoto continuava a chiedermi più o meno la stessa cosa, ho detto di poter sfruttare in qualche modo l'effetto Cherenkov ma non mi sembrava convintissimo, alla fine comunque ne sono uscito e volevano semplicemente che gli dicessi che metto due rilevatori (ammesso di averli trovati) a distanza x così con il primo posso misurare N0 e con il secondo N(x), beta lo calcolo conoscendo l'energia che avevo scelto io e quindi fissato x si può ricavare $\tau$. Infine mi hanno chiesto come scelgo x, bisogna assumere di conoscere l'ordine di grandezza di $\tau$ a priori e calcolare il valore di x$\sim \beta\gamma c\tau$ cioè il fattore per il quale N(x) si riduce di un fattore 1/e, in modo né da avere troppi pochi decadimenti, né che i muoni siano decaduti tutti, con l'energia scelta veniva x dell'ordine dei km, quindi mi hanno capire che 10 Gev erano troppi e conveniva usare energie minori.
        \item Mi hanno messo 29 nonostante un po' di incertezze e soprattutto una breve parte quasi di scena muta, sono abbastanza larghi io mi sarei dato 1 o 2 punti in  meno. La risposta alle domande spesso può essere più stupida di quanto uno non pensi quindi tanto vale dirla sempre se non è una cosa sbagliata. Spesso chiedono stime a priori e se poi dai calcoli esce fuori che sono sbagliate e uno capisce per quale motivo gli va bene comunque
    \end{enumerate}
\end{itemize}


\subsubsection{Carlo Mariani} 
\paragraph{Esperienze}
\begin{itemize}
    \item In genere fa tre domande: una sulla spettroscopia e sulle varie tecniche usate con confronti vari; un'altra in cui chiede sempre o quasi uno tra i due tipi di microscopi visti e un'altra che si allaccia al three step model, o comunque sull'ultima parte del programma (binding energy, etc.). Per quanto riguarda i dettagli delle lezioni di Scopigno e Postorino non entra molto in dettaglio. Ah ed è anche molto gettonata la domanda sull'oscillatore di Lorentz. E domandine varie su conduttori e semiconduttori.
\end{itemize}
 I have looked into my notes and I found a list of *some* questions that Mariani asked last year. I do not know how the program changed in detail, but however:
\begin{enumerate}
    \item STM: how it works, what kind of information it gives, what changes if one changes the polarisation of the tip with respect to the material;
    \item AFM: pretty much everything, the slides should be more than enough. On top of that, I remember that he asked me how to measure the cohesion force between the tip and the sample
    \item Synchrotron radiation: on what principle is it based on? How is the radiation distributed? How long is, in time, a pulse generated by a synchrotron (with calculations)? What is the spectral shape of the synchrotron radiation? What is the purpose of wigglers and undulators?
    \item Neutrons: How do they interact with matter? Can one use them for diffaction experiments? What kind of properties do they probe? What is the benefit of using neutrons for diffaction instead of electrons? Also, I remember that he also wants to know about inelastic neutron scattering: what can I probe with that technique? He ABSOLUTELY wants to know the energy needed to use neutrons for diffraction;
    \item Raman Scattering: describe the process, draw the diagram with energy levels and (maybe) the rate of interaction with the Fermi golden rule. What is the relationship between the Stokes and Anti-Stokes peaks? Often, he asks that indirectly by asking: can you use the raman effect to measure temperature? Why is the Anti-Stokes peak less intense than the Stokes peak?
    \item Photoemission spectroscopy: pretty much everything, he does that for a living! On what principle is it based? How does the process work (3 step aproximation) What is the difference between photoemission and photoabsorption (remember to tell him that the probed DOS is a joint DOS with the vacuum state, as it is the final state of the photoemitted electron)? Write the Fermi Golden rule for this process. What changes if I change the photon energy? What is the experimental setup of a photoemission experiment? What is the ARPES technique? What are the Auger peaks and why they do not change their position changing the photon energy? Can I know if an atom is chemically bound with a photoemission? (yes) What is the energy of a photoemitted electron and how does it affect its mean free path?(there is a graph on the slides, he wants you to sketch it)
    \item Photoabsorption: describe the process and write the Fermi Golden rule. What does it mean if a material has a direct or indirect gap and how does that affect the cross section (there should be the calculation somewhere in the slides). He wants you to sketch the band structure of some materials. He should have uploaded that chapter from Pastori Parravicini called something like "selected band structures". You need also to remember the gap of these materials (order of magnitude, if it is direct or not). Remember that indirect band transitions are aided by phonons.
    \item Drude and Drude-Lorentz models: compute the dielectric function, with all the calculations and draw the real and imaginary part of epsilon. Explain the difference between the models, what they describe and what are their limitations;
\end{enumerate}
\paragraph{gennaio 2022 - I appello}
\begin{itemize}
    \item STM (di che materiale è la punta? Che succede se inverto il bias?) 
\item Afm (come misuro la forza di coesione?) 
\item Fotoemissione ( tutto quello che abbiamo fatto. Importante il cammino libero medio degli elettroni) 
\item Xrays (energie in generale della diffrazione- bragg von laue) 
\item Raman ( spiegare i picchi di altezza differente tra stokes e antistokes, il rapporto ti permette di misurare la temperatura) 
\item Modello di Lorenz e Drude (bene bene, lo chiede spessissimo) 
\item Struttura delle bande (disegnare lo schema in LgammaX) del GaAS e degli elementi del 4 gruppo. 
\item Non ha mai chiesto LASER. Non chiede i dettagli degli esperimenti strani in genere (tipo con molecole biologiche o effetto condo o simili) 
LEED (Diffrazione con elettroni). Non sta chiedendo nulla di Saini tipo radiazione di sincrotrone 
\item There step model e regola d'oro di Fermi su Bardeen 
\item Cerchio di Ewald
\end{itemize}
\paragraph{9 o 10 febbraio 2022 - II appello}
\begin{enumerate}
    \item interazione radiazione materia in generale partendo da Maxwell; 
    \item andamento della funzione dielettrica e polarizzabilità complessa dal modello di Lorentz;
    \item afm;
    \item bande del grafene
\end{enumerate}
\begin{enumerate}
    \item modello di Lorentz, 
    \item come fare diffrazione con elettroni (energie, come "creare" elettroni, come accelerarli), 
    \item bande del germanio e dell'arsenuro di gallio
\end{enumerate}

\paragraph{gennaio 2023}
\begin{itemize}

\item intensità picchi diffrazione
\item sfera ewald
\item bande grafene: punti di sella, cono di dirac
\item fotoemissione: conservazione dell'energia
\item chemical shift Carbonio, esperimento fotoemissione
\item ARPES

\end{itemize}

\end{document}